Линии тока можно построить двумя способами. Можно построить поле скорости,
и загнать его в какой-нибудь TecPlot, но это черевато низкой точностью и дикими
спиралями. А можно воспользоваться функцией тока (она же --- векторный потенциал)
$\br\Psi: \rot\br\Psi = \br V$.
И тогда в двумере линии уровня $\lvert\br\Psi\rvert$ будут являться линиями тока.

\begin{equation*}
\Psi(\br r) = -\dfrac{1}{2 \pi} \sum_i \gamma_i \ln \left(\sqrt{(\br r - \br r_i)^2 + \delta^2} \right) = -\dfrac{1}{4\pi} \sum_i \gamma_i \ln \left({{\Delta r}^2 + \delta^2} \right) \text{--- вихрь.}
\end{equation*}

\begin{equation*}
\Psi(\br r) = -\dfrac{1}{2 \pi} \sum_i q_i \textrm{arctg} \left(\dfrac{\Delta r_y}{\Delta r_x} \right) \text{--- источник.}
\end{equation*}

\begin{equation*}
\Psi(\br r) = \br r \times \br V_\infty \text{--- бесконечность.}
\end{equation*}

